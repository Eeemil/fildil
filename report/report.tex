% Standard document setting, is the most appropriate output for reports that
% are to be printed. When handing in electronically please remove the
% "twoside" option.
\documentclass[10pt, a4paper]{article}

\usepackage[T1]{fontenc}
\usepackage[UKenglish]{isodate}
\usepackage[utf8]{inputenc}
\usepackage[english]{babel}
\usepackage{minted}

\usepackage{graphicx}
\usepackage{multicol}

% The following package allows us to define our own macros
% through "\LetLtxMacro\ourName\originalName"
\usepackage{letltxmacro}

% Adds vertical space between paragraphs, i.e. a clean linebreak
\usepackage{parskip}
\usepackage{verbatim}

% Math commands
\usepackage{amsmath}
\usepackage{amssymb}
\usepackage{amsthm}
\usepackage{amsfonts}
\newtheorem{definition}{Definition}
\usepackage{float}
\usepackage{footmisc} % Use symbols instead of numbers for footnotes
\usepackage{csquotes}
\usepackage{hyperref}
\hypersetup{
  colorlinks = true, % Colours links instead of ugly boxes
    urlcolor = blue, % Colour for external hyperlinks
      linkcolor = blue, % Colour of internal links
        citecolor = red % Colour of citations
	}

% Used to get the last page of the document, used in our document headers
\usepackage{lastpage}

% Specify date format
\usepackage[yyyymmdd]{datetime}
\renewcommand\dateseparator{-} % Separate date tokens with a "-"

% Use fancy document headers
\usepackage{fancyhdr}

\usepackage{letltxmacro}
\usepackage{float}

% An environment for nice looking quotes
\makeatletter
\newenvironment{chapquote}[2][2em]
{\setlength{\@tempdima}{#1}%
  \def\chapquote@author{#2}%
    \parshape 1 \@tempdima \dimexpr0.5\textwidth-2\@tempdima\relax}
    {\par\normalfont\hfill\ \chapquote@author\hspace*{\@tempdima}\par}
    \makeatother

%
% BibLaTeX
%
% Great reference system
% Documentation:
% http://mirrors.ctan.org/macros/latex/contrib/biblatex/doc/biblatex.pdf
% ___________________________________________________________
\usepackage[style = ieee, urldate =comp, backend=biber]{biblatex}
\addbibresource{sources.bib}

\newcommand{\department}{Department of Computing Science}
\newcommand{\uni}{Ume\aa\ University}

\newcommand{\authors}{Emil Marklund (\texttt{eeemil@cs.umu.se})\\Timmy Olsson(\texttt{c12ton@cs.umu.se})}
  \newcommand{\documentTitle}{Project report -- Fildil}
  \title{\documentTitle}
  \author{\authors}
  \newcommand{\coursename}{Advanced distributed systems}
  \newcommand{\coursecode}{5DV121}
  \newcommand{\semester}{HT16}
  \newcommand{\credits}{7.5}
  \newcommand{\instructors}{P-O Östberg (\texttt{p-o@cs.umu.se}), Adam Dahlgren (\texttt{dali@cs.umu.se})}
  \pagestyle{fancy}

\usepackage{color}
\usepackage{float}
\usepackage{fancyvrb}
\usepackage[scaled]{beramono}

\usepackage{listings}

% Code snippet syntax
\lstset{backgroundcolor=\color{GhostWhite}, basicstyle={\small\ttfamily},
  breakatwhitespace=false, breaklines=false, captionpos=b,
  commentstyle=\color{Green}, frame=single, keywordstyle=\color{Blue},
  language=C, % C has somewhat similar syntax to Rust
  morekeywords={fn, let, mut, unsafe, trait, as, macro, override, ref, type,
    virtual, box, in, mod, priv, typeof,
    where}, % Additional keywords used in Rust
  rulecolor=\color{LightSteelBlue}, showspaces=false, showtabs=false,
  showstringspaces=false, stringstyle=\color{FireBrick}, tabsize=4, float }
\renewcommand{\lstlistingname}{Code snippet}

% \BeforeBeginEnvironment{listings}{\par\noindent\begin{minipage}{\linewidth}}
%   \AfterEndEnvironment{listings}{\end{minipage}\par\addvspace{\topskip}}

\begin{document}

\raggedbottom

\date{}
\begin{titlepage}
  \maketitle

   \fancyfoot{}

  \thispagestyle{fancy}
    \headheight 35pt

  \lhead{\small \department \\
      \uni} % These two are also defined in config.sty

  \rhead{\small \today} %date



  \cfoot{\coursename\ (\coursecode), \credits\ hp -- \semester\\
      Supervisor(s): \instructors}

\end{titlepage}

% Header settings
\fancyhead[LE,RO]{\thepage(\pageref{LastPage})}
\fancyheadoffset[LE,RO]{12mm}
\fancyhead[LO,RE]{\coursename}
\fancyfoot[L,R,C]{}

\tableofcontents

\clearpage

\pagenumbering{arabic}

\setlength{\parindent}{0pt}

\section{Introduction}

This document is a project report for the development of \emph{Fildil}, a video
streaming application designed to be scalable by splitting load between everyone
interested in the stream in a peer to peer manner.

This section contains a number of subsections: in \autoref{sec:goals} you may
read more extensible about the problems Fildil is supposed to solve,
\autoref{sec:terminology} contains a definite guide to the terminology used
throughout this report and \autoref{sec:building-running} consists of
instructions on using the application.

\autoref{sec:design} presents the high-level design philosophies for building
Fildil, e.g. algorithms. In \autoref{sec:system}, the actual implementation is
discussed in greater detail. Finally, the performance of Fildil is evaluated in
\autoref{sec:results}. This document is ended with \autoref{sec:discussion}
which presents our conclusions, improvements and experiences from the project.

\subsection{Terminology}
\label{sec:terminology}

The following terminology is used throughout the report.

\begin{description}
\item[Video source] the source used to generate the streaming data. For example,
  \texttt{FFmpeg} reading a video file could be a video source.
\item[Chunk] A data packet containing a small part of the streamed
  content, with metadata.
\item[Node] Any participant of the \emph{streaming network}: downloading and/or
  uploading \emph{chunks} of the stream
\item[Streaming network] All \emph{nodes} participating in generating and
  sharing chunks.
\item[Primary node] \emph{Node} reading directly from the \emph{video source},
  \emph{generating} chunks for the \emph{streaming network}.
\item[Peer node] \emph{Node} reading from the \emph{streaming network}, sharing
  \emph{chunks} with other nodes. Differs from \emph{primary node} in that it
  does not generate any chunks on its own, it merely downloads chunks and
  uploads them to other \emph{peer nodes} on request.
\end{description}

\subsection{Goals}
\label{sec:goals}

The application has been designed with the following goals in mind:

\begin{description}
\item[Robustness] The system should be able to \emph{automatically} handle
  increase of nodes(i.e computers collaborating on providing the service).
\item[Extensibility] To the extent it is possible, new features should be able
  to be added to Fildil without having to redesign earlier components.
\item[Load balancing/consistent QoS] Fildil should be able to adapt to spikes in
  load in order to provide a consistent Quality of Service.
\item [Multiple streamer and viewer clients] The system should support multiple
  streaming clients as well as multiple viewer clients.
\end{description}

In Fildil, every node joining the streaming network polls one (or more) existing
nodes to get a list of the existing node's peers. In that manner, a joining node
may recursively poll peers within the streaming network to build knowledge of
the streaming network. Simultaneously, the polled peers gets knowledge of a
joining node. 

This automatic organization scheme has been implemented in accordance with the
\emph{robustness}-goal, and the \emph{multiple streamer and viewer client}
goal.

Google Protocol Buffers has been used for designing the service with
GRPC. However, with the existing protocol buffer and the operations designed for
the service, it would be possible to implement Fildil in other languages without
too much effort. This is in accordance with the \emph{extensibility}-goal.

Fildil splits the video source into chunks, and the complete streaming network
shares the responsibility of distributing the chunks. This is thought to fulfill
the \emph{load balancing/consistent QoS} goal.

\subsection{Building \& running}
\label{sec:building-running}

\subsubsection{Prerequisites}

The following prerequisites exists for building and running Fildil:

\begin{description}
\item[Java 1.8] Fildil is written in Java and makes use of features available
  only in Java 8 or newer. Throughout development, \texttt{Java Runtime
    Environment 1.8.0\_25-b17} has been used.
\item[Maven] In order to build Fildil, Maven is required. \texttt{Maven 3.0.5}
  has been used during development.
\item[FFmpeg] Fildil uses FFmpeg for reading a video source. \texttt{FFmpeg
    3.2-2~bpo8+2, built with gcc 4.9.2 (Debian 4.9.2-10)} has been used
  during testing. \textbf{FFmpeg is only a prerequisite for primary nodes}.
\item[POSIX-compliant OS] Pipes are used for communication between Fildil and
  other programs (such as \emph{VLC} and \emph{FFmpeg}). Testing has been
  performed on \texttt{Debian 3.16.36-1+deb8u2 (Linux 3.16.0-4-amd64)}.
\item[VLC media player] In order to watch the streamed content, a media player
  able to manage streaming needs to be used. \texttt{VLC media player 2.2.4
    (revision 2.2.3-37-g888b7e89)} has been used during testing.
\end{description}

\subsubsection{Compiling Fildil}

mvn compile

\subsubsection{Running a primary node}

Run,

\subsubsection{Running a peer node}

Forrest, run!

\section{Design: decentralized peer to peer streaming}
\label{sec:design}

Hello

\section{System architecture}
\label{sec:system}

It was built by using code

\subsection{Managing errors}

\subsubsection{Node crash}
%Todo: Expand on this
If a node crashes, all information about that node should be discarded from all
other nodes. This is easy enough if a node that crashes never comes
back. However, problems can occur if the following situation comes up:

Say that Node A has highest chunk 1000 and address ``1.2.3.4:5678''. Node A
crashes and comes back up instantly, but losing all downloaded chunks in the
process. The node will then have a new highest chunk value of -1. However, this
can impair the streaming logic of other nodes assuming that Node A has highest
chunk value of 1000.

This is solved by assigning a UUID to each node on startup. Nodes communicating
with Node A will see that Node A has changed UUID, and will then dismiss all old
information about Node A.

\section{Results}
\label{sec:results}

The results are in

\subsection{Benchmarks}

Fildil scores over 9000

\section{Discussion}
\label{sec:discussion}

\subsection{Improvements}

\subsubsection{Data storage}
\label{sec:storage-improvements}

In the current implementation of Fildil, the complete stream is downloaded from
the beginning of the stream, to the end. Also, it is kept in memory for all node
types. This may be impractical for long-running streams, where the stream could
be very large in size:

\begin{enumerate}
\item Keeping lots of video streaming data in memory is very costly with the
  computer hardware of today. There is a need of caching the data for any node
  storing large parts of the stream. Caching has not been implemented in Fildil,
  instead, we have only used streams that have been able to be kept in memory on
  our testing equipment.
\item Another way to deal with the large data size in a peer to peer network is
  to split the streaming content storage on the nodes such that no node needs to
  keep a large part by itself. This mechanism has been left out due to time
  constraints. %TODO: has it still been left out?
\item Downloading the stream from beginning to end is probably not a realistic
  behavior for live streams. A more realistic behavior is to start the
  downloading at the most recent point of the stream (as close to ``live'' as
  possible). This would reduce the data size needed. This has been left out in
  Fildil, as we have argued that this behavior would make benchmarking harder
  and add complexity to our test environment, while only adding the benefit of
  needing to store less data in memory.
\end{enumerate}

\newpage

% \printbibliography
%\begin{thebibliography}{99}

%\end{thebibliography}

\end{document}

